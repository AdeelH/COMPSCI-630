\documentclass[twoside]{article}
\setlength{\oddsidemargin}{0.25 in}
\setlength{\evensidemargin}{-0.25 in}
\setlength{\topmargin}{-0.6 in}
\setlength{\textwidth}{6.5 in}
\setlength{\textheight}{8.5 in}
\setlength{\headsep}{0.75 in}
\setlength{\parindent}{0 in}
\setlength{\parskip}{0.1 in}

\usepackage{graphicx}
\usepackage{url}
\usepackage{tikz}
\usetikzlibrary{positioning}
\usepackage{float}
%
% The following commands sets up the lecnum (lecture number)
% counter and make various numbering schemes work relative
% to the lecture number.
%
\newcounter{lecnum}
\renewcommand{\thepage}{\thelecnum-\arabic{page}}
\renewcommand{\thesection}{\thelecnum.\arabic{section}}
\renewcommand{\theequation}{\thelecnum.\arabic{equation}}
\renewcommand{\thefigure}{\thelecnum.\arabic{figure}}
\renewcommand{\thetable}{\thelecnum.\arabic{table}}
\newcommand{\dnl}{\mbox{}\par}

%
% The following macro is used to generate the header.
%
\newcommand{\lecture}[4]{
  \pagestyle{myheadings}
  \thispagestyle{plain}
  \newpage
  \setcounter{lecnum}{#1}
  \setcounter{page}{1}
  \noindent
  \begin{center}
  \framebox{
     \vbox{\vspace{2mm}
   \hbox to 6.28in { {\bf COMPSCI~630~~~Systems
                       \hfill Spring 2018} }
      \vspace{4mm}
      \hbox to 6.28in { {\Large \hfill Lecture #1: #2  \hfill} }
      \vspace{2mm}
      \hbox to 6.28in { {\it Lecturer: #3 \hfill Scribe(s): #4} }
     \vspace{2mm}}
  }
  \end{center}
  \markboth{Lecture {#1}: #2}{Lecture {#1}: #2}
  \vspace*{4mm}
}

%
% Convention for citations is authors' initials followed by the year.
% For example, to cite a paper by Leighton and Maggs you would type
% \cite{LM89}, and to cite a paper by Strassen you would type \cite{S69}.
% (To avoid bibliography problems, for now we redefine the \cite command.)
%
\renewcommand{\cite}[1]{[#1]}

% \input{epsf}

%Use this command for a figure; it puts a figure in wherever you want it.
%usage: \fig{NUMBER}{FIGURE-SIZE}{CAPTION}{FILENAME}
\newcommand{\fig}[4]{
           \vspace{0.2 in}
           \setlength{\epsfxsize}{#2}
           \centerline{\epsfbox{#4}}
           \begin{center}
           Figure \thelecnum.#1:~#3
           \end{center}
   }

% Use these for theorems, lemmas, proofs, etc.
\newtheorem{theorem}{Theorem}[lecnum]
\newtheorem{lemma}[theorem]{Lemma}
\newtheorem{proposition}[theorem]{Proposition}
\newtheorem{claim}[theorem]{Claim}
\newtheorem{corollary}[theorem]{Corollary}
\newtheorem{definition}[theorem]{Definition}
\newenvironment{proof}{{\bf Proof:}}{\hfill\rule{2mm}{2mm}}

% Some useful equation alignment commands, borrowed from TeX
\makeatletter
\def\eqalign#1{\,\vcenter{\openup\jot\m@th
 \ialign{\strut\hfil$\displaystyle{##}$&$\displaystyle{{}##}$\hfil
     \crcr#1\crcr}}\,}
\def\eqalignno#1{\displ@y \tabskip\@centering
 \halign to\displaywidth{\hfil$\displaystyle{##}$\tabskip\z@skip
   &$\displaystyle{{}##}$\hfil\tabskip\@centering
   &\llap{$##$}\tabskip\z@skip\crcr
   #1\crcr}}
\def\leqalignno#1{\displ@y \tabskip\@centering
 \halign to\displaywidth{\hfil$\displaystyle{##}$\tabskip\z@skip
   &$\displaystyle{{}##}$\hfil\tabskip\@centering
   &\kern-\displaywidth\rlap{$##$}\tabskip\displaywidth\crcr
   #1\crcr}}
\makeatother

% **** IF YOU WANT TO DEFINE ADDITIONAL MACROS FOR YOURSELF, PUT THEM HERE:



% Some general latex examples and examples making use of the
% macros follow.

\begin{document}
\lecture{20}{April 23}{Emery Berger}{Abhinav Jangda}
%FILL IN THE RIGHT INFO.
%\lecture{**LECTURE-NUMBER**}{**DATE**}{**LECTURER**}{**SCRIBE**}
\section{Clock Synchronization}
CPU contains an internal clock, which is powered by a quartz crystal. 
These clocks have a clock drift of half a second per day. Hence, within
30 days a quartz clock will be 15 seconds ahead or behind the actual clock.
These clocks have to be synchronized after some time with a time server.
Time servers contains atomic clocks which are several magnitudes accurate
than quartz clock. To do synchronization protocols like Network Time Protocol (NTP)
is used. However, synchronization by NTP suffers due to problems like
network delays. Hence, using absolute time is not meaningful in distributed
systems.

\section{Happens before relation}
Happens before relation defines the notion of causality between events.
For example, if process P$_1$ sends $a$ to process P$_2$ and due to
$a$, P$_2$ sends $b$ to P$_3$, then we say that $a$ has happened before
$b$. In other words, $a$ is the cause of $b$. 
\begin{figure}[H]
\begin{tikzpicture}
\node (A) {A};
\node [above right=of A] (B) {B};
\node [right=of B] (C) {C};
\node [below right=of A] (X) {X};
\node [right=of X] (Y) {Y};
\node [below right=of C] (Z) {Z};
\draw [->] (A) -- (B);
\draw [->] (B) -- (C);
\draw[->] (C) edge (Z);
\draw [->] (A) -- (X);
\draw [->] (X) -- (Y);
\draw [->] (Y) -- (Z);
\end{tikzpicture}
\caption{Example of Happens before relation\label{fig:happens-before-ex}}
\end{figure}

Figure~\ref{fig:happens-before-ex} shows an example of causality between 
events \texttt{A}, \texttt{B}, \texttt{C}, \texttt{X}, \texttt{Y}, \texttt{Z}. 
\texttt{A} happens before every other event. \texttt{B} happens before \texttt{C}
and \texttt{Z}. However, we cannot say if \texttt{X} happens before \texttt{B}
or not. Hence, happens before relation (also called causal relationship)
provides us with partial order but not total order.

\section{Lamport Clock}
Lamport Clock is an implementation of causality. Every machine will have
one global integer which is incremented whenever an event has happened on that 
machine, which is Lamport Clock (\texttt{LC}) value for that particular machine.
Whenever, machine $A$ sends a message $m$ to machine $B$, then machine
$A$ will also add its \texttt{LC} value to the message. Machine $B$ will then
update its \texttt{LC} value as the $B.LC \gets max (m.LC, B.LC) + 1$. 
The Lamport Clock value for a machine defines the number of messages it has
received. Now, we can see that for any two messages $m$ and $n$, if 
$m$ has happened before $n$, then $m.LC < n.LC$. However, we cannot say
that if $m.LC < n.LC$ then $m$ has happened before $n$. Hence, we can only
achieve partial ordering but not total ordering with Lamport Clock.
To achieve Total Ordering we can use Vector Clocks or Totally ordered multicast
using Lamport Clock.

\section{Race Detection}
A race condition happens when an application depends on the sequence or timing
of processes or threads for it to operate properly. In terms of causality,
race condition is an absence of a happen before relation. A benign data-race
is an intentional data-race whose existence does not affect the correctness 
of the program. These data-races may help in getting better performance. 

Race detection can be performed statically and dynamically. In order to
do static race
detection, the tool has to identify reads and writes and establish 
happens before relationship between them. Static race
detection is difficult for compiler to do. In general static analysis is
limited by halting problem. In languages using pointers and references,
alias analysis of all heap variables (including locks) increases the 
difficulty of doing static analysis.

Race detection is termed sound, if it says that if there is no race
then there is definitely no race in the program. In this case, however, 
it may be possible that the program analysis gives false positives, i.e.,
reporting a race condition even if it isn't. A program analysis is complete
if it has no false positives, i.e., if it reports a race then the race is real.

\begin{figure}[ht]
\begin{minipage}[b]{0.45\linewidth}
\centering
\textbf{Dynamic Race Detection}\\
\begin{itemize}
\item Sees the execution trace and code of the program
\item Input dependent
\item No false positives
\end{itemize}
%\caption{default}
%\label{fig:figure1}
\end{minipage}
\hfill
\begin{minipage}[b]{0.45\linewidth}
\centering
\textbf{Static Race Detection}\\
\begin{itemize}
\item Can only see the code of the program
\item Input independent
\item Can have false positives
\item Returns facts regarding program
\end{itemize}
%\includegraphics[width=\textwidth]{filename2}
%\caption{default}
%\label{fig:figure2}
\end{minipage}
\end{figure}

There are two types of dynamic race detection techniques: (i) lockset
analysis, and (ii) happen before analysis. In Lockset Analysis, there is
a set of locks for every variable. During the beginning of a program,
all the sets are universal set, i.e., lock sets of all variables contains
all locks. A set of current locks is maintained 
for each thread. Whenever lock is obtained, this lock is added to the
set of current locks and whenever lock is released, this lock is removed
from the set of current locks. Whenever a variable is accessed, it's
lockset is updated as the intersection of current lockset and variable's
lockset. If the intersection is empty, then this is a potential race condition.
Unfortunately, Lockset analysis is expensive and can give false positives.

Happen-Before analysis uses vector clocks. It is based on the idea that
certian operations define an ordering between operations of threads and if
a variable access can't establish itself as being after the previous one,
then we have detected an actual race. Each variable stores the thread
clock value for its recent access of each thread. When an access occur,
then this access must have a Lamport clock value more than the Lamport clock value
of previous thread.

\end{document}
