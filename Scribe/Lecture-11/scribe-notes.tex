\documentclass[twoside]{article}
\setlength{\oddsidemargin}{0.25 in}
\setlength{\evensidemargin}{-0.25 in}
\setlength{\topmargin}{-0.6 in}
\setlength{\textwidth}{6.5 in}
\setlength{\textheight}{8.5 in}
\setlength{\headsep}{0.75 in}
\setlength{\parindent}{0 in}
\setlength{\parskip}{0.1 in}

\usepackage{graphicx}
\usepackage{url}

%
% The following commands sets up the lecnum (lecture number)
% counter and make various numbering schemes work relative
% to the lecture number.
%
\newcounter{lecnum}
\renewcommand{\thepage}{\thelecnum-\arabic{page}}
\renewcommand{\thesection}{\thelecnum.\arabic{section}}
\renewcommand{\theequation}{\thelecnum.\arabic{equation}}
\renewcommand{\thefigure}{\thelecnum.\arabic{figure}}
\renewcommand{\thetable}{\thelecnum.\arabic{table}}
\newcommand{\dnl}{\mbox{}\par}

%
% The following macro is used to generate the header.
%
\newcommand{\lecture}[4]{
  \pagestyle{myheadings}
  \thispagestyle{plain}
  \newpage
  \setcounter{lecnum}{#1}
  \setcounter{page}{1}
  \noindent
  \begin{center}
  \framebox{
     \vbox{\vspace{2mm}
   \hbox to 6.28in { {\bf CMPSCI~630~~~ Systems
                       \hfill Fall 2014} }
      \vspace{4mm}
      \hbox to 6.28in { {\Large \hfill Lecture #1  \hfill} }
%       \hbox to 6.28in { {\Large \hfill Lecture #1: #2  \hfill} }
      \vspace{2mm}
      \hbox to 6.28in { {\it Lecturer: #3 \hfill Scribe: #4} }
     \vspace{2mm}}
  }
  \end{center}
  \markboth{Lecture #1: #2}{Lecture #1: #2}
  \vspace*{4mm}
}

%
% Convention for citations is authors' initials followed by the year.
% For example, to cite a paper by Leighton and Maggs you would type
% \cite{LM89}, and to cite a paper by Strassen you would type \cite{S69}.
% (To avoid bibliography problems, for now we redefine the \cite command.)
%
\renewcommand{\cite}[1]{[#1]}

% \input{epsf}

%Use this command for a figure; it puts a figure in wherever you want it.
%usage: \fig{NUMBER}{FIGURE-SIZE}{CAPTION}{FILENAME}
\newcommand{\fig}[4]{
           \vspace{0.2 in}
           \setlength{\epsfxsize}{#2}
           \centerline{\epsfbox{#4}}
           \begin{center}
           Figure \thelecnum.#1:~#3
           \end{center}
   }

% Use these for theorems, lemmas, proofs, etc.
\newtheorem{theorem}{Theorem}[lecnum]
\newtheorem{lemma}[theorem]{Lemma}
\newtheorem{proposition}[theorem]{Proposition}
\newtheorem{claim}[theorem]{Claim}
\newtheorem{corollary}[theorem]{Corollary}
\newtheorem{definition}[theorem]{Definition}
\newenvironment{proof}{{\bf Proof:}}{\hfill\rule{2mm}{2mm}}

% Some useful equation alignment commands, borrowed from TeX
\makeatletter
\def\eqalign#1{\,\vcenter{\openup\jot\m@th
 \ialign{\strut\hfil$\displaystyle{##}$&$\displaystyle{{}##}$\hfil
     \crcr#1\crcr}}\,}
\def\eqalignno#1{\displ@y \tabskip\@centering
 \halign to\displaywidth{\hfil$\displaystyle{##}$\tabskip\z@skip
   &$\displaystyle{{}##}$\hfil\tabskip\@centering
   &\llap{$##$}\tabskip\z@skip\crcr
   #1\crcr}}
\def\leqalignno#1{\displ@y \tabskip\@centering
 \halign to\displaywidth{\hfil$\displaystyle{##}$\tabskip\z@skip
   &$\displaystyle{{}##}$\hfil\tabskip\@centering
   &\kern-\displaywidth\rlap{$##$}\tabskip\displaywidth\crcr
   #1\crcr}}
\makeatother

% **** IF YOU WANT TO DEFINE ADDITIONAL MACROS FOR YOURSELF, PUT THEM HERE:



% Some general latex examples and examples making use of the
% macros follow.

\begin{document}

%FILL IN THE RIGHT INFO.
%\lecture{**LECTURE-NUMBER**}{**DATE**}{**LECTURER**}{**SCRIBE**}
\lecture{11}{October 9}{Emery Berger}{Nicholas Braga, Cibele Freire}

\section{Introduction}

TODO

\section{Confinement}

During his Turing Award recipient lecture \textit{Reflections on Trusting Trust}, Ken Thompson outlined how he was able to inject a 'virus' into a compiler that effectively served as a backdoor. While it's possible that by the time Thompson gave this speech, the compiler described might haven no longer been vulnerable, it does highlight the context in which it was given: Security was very much a \textit{hypothetical} concern.

\subsection{Channels}
 
It is possible for data to leak through two types of channels:

\begin{itemize}
\item \textbf{Overt}: An explicit flow of information 
\item \textbf{Covert}: An implicit flow of information
\end{itemize}

Examples of exploiting information have been widely demonstrated and include packet traffic analysis, and timing attacks (using the time that a program with logical flow takes to execute, in order to predict which branches it has followed). Even power consumption can be used to predict a user's web traffic (shown by Brian Levine and Kevin Fu).
At the time Thompson gave his speech, many in the audience were likely still skeptical that covert traffic could be exploited in this way.

\subsection{'Timing Attacks are Real'}

This paper revealed that passwords could be intercepted over a network. This was done by building a model of key travel distance; only a few observations (from packets) were enough to make a prediction. In general, network traffic patterns can be used to infer a user's browsing activity. Using \texttt{promiscuous mode} (where all network traffic - not just that which is addressed to your MAC address - is process by your network interface). \textit{[Aside: Technically, all wireless traffic is interceptable due to being radio waves, and even using unencrypted headers]} These types of attacks can be battled with nonce characters and/or pseudorandom packet interarrival times.

This however leads to an important trade-off: \textbf{Latency vs. prviacy}. \textit{Low} latency applications leak timing information, since their activity streams can be reconstructed by an observer with high certainity. \textit{High} latency applications have the advantage of being able to shuffle and delay activity (an example being mixes as used by \textit{Tor}.
Many attacks against systems are simply due to poor system design (Examples of poor/hard-to-prevent exploits include over-the-shoulder and fingerprint reading for smartphone users, and the use of infrared of ATM keypads). Above even these exploits, which are mostly not the fault of developers, is rubber-hose cryptography: the forced handover of credentials (or assault with a rubber hose...).

\subsection{Formal Framework}

TODO

\subsection{Levels of Confinement}

\begin{itemize}
\item \textbf{OS/VM level} is made possible through \textit{sandboxing}. One way of using this is in browsers, where some separate 'tabs' into processes.
\item \textbf{Filesystem level} can be implemented using \texttt{chroot jail}. This tells the OS that for the given process, \texttt{root} should be given as some higher-level directory.
\item \textbf{Programming Language/Binary level} can be implemented using \textsc{SFI}, where one instruments read/write calls with a reference monitor. This enables software fault isolation.
\end{itemize}

\section{Tainting}

TODO Issues with tainting...

\section{Dynamic Information Flow Tracking (DIFT)}

This hardware technique enforces that \texttt{allow} information through all flows of the program. This constraint results in \textsc{DIFT} being too slow to be practical for high-performance languages.
An implementation \textsc{DIFT} is \textsc{JIF} (Java with Information Flow) by Andrew Myers, which used static information flow control.

\section{Noninterference}

The noninterference property states that \textit{low-value and high-value data should not interfere}.
This is a desirable property to preserve, and one that is only vulnerable in covert channels. An example of a noninterference exploit can be imagined in the browser through exfiltration to track a user. (Or, JavaScript could maliciously download at will.) 


\end{document}
