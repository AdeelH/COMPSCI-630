\documentclass[twoside]{article}
\setlength{\oddsidemargin}{0.25 in}
\setlength{\evensidemargin}{-0.25 in}
\setlength{\topmargin}{-0.6 in}
\setlength{\textwidth}{6.5 in}
\setlength{\textheight}{8.5 in}
\setlength{\headsep}{0.75 in}
\setlength{\parindent}{0 in}
\setlength{\parskip}{0.1 in}

\usepackage{graphicx}
\usepackage{url}

%
% The following commands sets up the lecnum (lecture number)
% counter and make various numbering schemes work relative
% to the lecture number.
%
\newcounter{lecnum}
\renewcommand{\thepage}{\thelecnum-\arabic{page}}
\renewcommand{\thesection}{\thelecnum.\arabic{section}}
\renewcommand{\theequation}{\thelecnum.\arabic{equation}}
\renewcommand{\thefigure}{\thelecnum.\arabic{figure}}
\renewcommand{\thetable}{\thelecnum.\arabic{table}}
\newcommand{\dnl}{\mbox{}\par}

%
% The following macro is used to generate the header.
%
\newcommand{\lecture}[4]{
  \pagestyle{myheadings}
  \thispagestyle{plain}
  \newpage
  \setcounter{lecnum}{#1}
  \setcounter{page}{1}
  \noindent
  \begin{center}
  \framebox{
     \vbox{\vspace{2mm}
   \hbox to 6.28in { {\bf CMPSCI~630~~~Systems
                       \hfill Fall 2014} }
      \vspace{4mm}
      \hbox to 6.28in { {\Large \hfill Lecture #1  \hfill} }
%       \hbox to 6.28in { {\Large \hfill Lecture #1: #2  \hfill} }
      \vspace{2mm}
      \hbox to 6.28in { {\it Lecturer: #3 \hfill Scribe: #4} }
     \vspace{2mm}}
  }
  \end{center}
  \markboth{Lecture #1: #2}{Lecture #1: #2}
  \vspace*{4mm}
}

%
% Convention for citations is authors' initials followed by the year.
% For example, to cite a paper by Leighton and Maggs you would type
% \cite{LM89}, and to cite a paper by Strassen you would type \cite{S69}.
% (To avoid bibliography problems, for now we redefine the \cite command.)
%
\renewcommand{\cite}[1]{[#1]}

% \input{epsf}

%Use this command for a figure; it puts a figure in wherever you want it.
%usage: \fig{NUMBER}{FIGURE-SIZE}{CAPTION}{FILENAME}
\newcommand{\fig}[4]{
           \vspace{0.2 in}
           \setlength{\epsfxsize}{#2}
           \centerline{\epsfbox{#4}}
           \begin{center}
           Figure \thelecnum.#1:~#3
           \end{center}
   }

% Use these for theorems, lemmas, proofs, etc.
\newtheorem{theorem}{Theorem}[lecnum]
\newtheorem{lemma}[theorem]{Lemma}
\newtheorem{proposition}[theorem]{Proposition}
\newtheorem{claim}[theorem]{Claim}
\newtheorem{corollary}[theorem]{Corollary}
\newtheorem{definition}[theorem]{Definition}
\newenvironment{proof}{{\bf Proof:}}{\hfill\rule{2mm}{2mm}}

% Some useful equation alignment commands, borrowed from TeX
\makeatletter
\def\eqalign#1{\,\vcenter{\openup\jot\m@th
 \ialign{\strut\hfil$\displaystyle{##}$&$\displaystyle{{}##}$\hfil
     \crcr#1\crcr}}\,}
\def\eqalignno#1{\displ@y \tabskip\@centering
 \halign to\displaywidth{\hfil$\displaystyle{##}$\tabskip\z@skip
   &$\displaystyle{{}##}$\hfil\tabskip\@centering
   &\llap{$##$}\tabskip\z@skip\crcr
   #1\crcr}}
\def\leqalignno#1{\displ@y \tabskip\@centering
 \halign to\displaywidth{\hfil$\displaystyle{##}$\tabskip\z@skip
   &$\displaystyle{{}##}$\hfil\tabskip\@centering
   &\kern-\displaywidth\rlap{$##$}\tabskip\displaywidth\crcr
   #1\crcr}}
\makeatother

% **** IF YOU WANT TO DEFINE ADDITIONAL MACROS FOR YOURSELF, PUT THEM HERE:



% Some general latex examples and examples making use of the
% macros follow.

\begin{document}

%FILL IN THE RIGHT INFO.
%\lecture{**LECTURE-NUMBER**}{**DATE**}{**LECTURER**}{**SCRIBE**}
\lecture{18}{November 13}{Emery Berger}{Nicholas Braga, Teddy Sudol}

\section{Ordering of Events in Distributed Systems}

In a distributed system, there is no total ordering without a global, synchronized, physical clock for each process. Additionally, it is not possible to establish a totally correct global time in a distributed system. For this reason, Leslie Lamport invented \textbf{Logical (Lamport) clocks}.

\subsection{Lamport Clocks}

\textbf{Lamport clocks} are a scalar that enable one to reason on distributed events using the \textit{happens-before} (HB) relation. 
% TODO: process timeline

\begin{equation}
    C(e_{1}) < C(e_{2}) \Rightarrow e_{1} HB  e_{2}
\end{equation}

The use of \[<\] rather than \[\leq\] is a "hack" which enables us to more easily reason on uninitalized clock values. 

This, however, is not fine-grained enough, since it is possible to lose information from this model. In particular, logical clocks cause an overapproximation on the number of events that you are able obtain a relation on. 

\subsection{Vector Clocks}

\textbf{Vector clocks} are a generalization of Lamport clocks. Broadly, a vector clock keeps a view on all other clocks in the system \textit{relative to their last interaction} (if at all).
% TODO: Vector clock process timeline
Each vector clock maintains \texttt{n} cells (corresponding to \texttt{n} other processes). Each cell contains the value of the process \texttt{i} total number of sends when these processes last interacted. If they have not interacted yet, this value (remains) initialized as \texttt{-1}. Thus, the vector (corresponding to process) \texttt{i} maintains for process \texttt{j}
\begin{equation}
    where \, i \neq j \, \forall \, n \, processes \; \frac{count\, of\, total\, sends\, for\, process\, i}{MAX\,(\,count\, of\, total\, sends\, of\, i\, last\, observed\,)}
\end{equation}

% NB: I was just throwing down the rest of my notes in case they were useful when you merge, feel free to edit/remove

Because everything is being updated in the same order, every process can have a \textbf{consistent view}. Some issues with this, however, are the size of vector clocks needs to be initialized at the outset of sync. In the case of failure, with logging, we can rollback all processes and replay (locally, we cannot assume idempotence; appears as a delay to the outside).

All of these concepts can also be applied to a shared memory system.

Bohr Bug: deterministic bug; "easy to find", not necessarily easy to fix but gives more context; very common.

Heisenbug: nondeterministic; hard to fix, but does not happen as frequently.

Jim Gray was chiefly concerned with fault tolerance and availability.

False positive problem: a fix not needed
False negative problem: a fix needed but not given

Concurrency errors are a canonical Heisenbug; including, timing (ordering of events -- schedular: nondeterminism of locks, etc.) network I/O, logging time, random number generators, crypography, etc.

Monte Carlo algorithm: using randomness to compute a difficult problem ("The more trials, the more accurate your algorithm will be")

Von Neumann: "Original Sin" : Believing true randomness can come from a deterministic function


\end{document}
